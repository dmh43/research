\documentclass{article}

\begin{document}

\title{Ranking: Consistency and Rates}
\author{Dany Haddad}
\date{July 2020}
\maketitle

\section{Introduction}

This is a work in progress document discussing the consistency of
common convex surrogate functions for ranking as well as the
convergence rate of the corresponding empirical risk. First we
summarize the main points of Duchi et al. in this
regard~\cite{duchi-2010-ranking,duchi-2013-ranking}. Then we discuss
some relevant research directions, in partictular, we are interested
in achieving fast convergence rates for ranking
applications.
\footnote{I also point out confusion and questions I have in the footnotes.}

\section{Literature Review}

\subsection{The Asymptotics of Ranking Algorithms}

\subsubsection{Overview}

The authors first introduce the ranking problem and describe a
pairwise loss function, equivalent to the pairwise 0-1~ranking
loss. They then define several forms of consistency for ranking
problems and unify them under some conditions
(theorem~1). Surprisingly, common surrogates for the pairwise
0-1~ranking loss function are shown to not be consistent.

The next section discusses how aggregating the pairwise preferences
into lists allows us to define consistent and tractable surrogate loss
functions. The key idea is that the surrogates must be order preserving.
\footnote{The authors describe inconsistency as arising due to a lack of complete preference information. This point is not entirely clear to me since the proof of inconsistency seems to depend on the variability in preference information rather than the lack of information. See \cite{ravikumar-2011-ranking} for a discussion on how a lack of normalization leads to ``non-robustness''.}

Given that our new surrogate loss functions consist of an aggregation
step, it is natural to consider the aggregation in batches. In
particular, the authors describe the loss function as a sum of
U-statistics. However, this is seems mostly to simplify the analysis
rather than to leverage the fact that U-statistics are UMVUEs.
\footnote{The authors mention that they introduce the U-statistics
  based risk functional to be able to analyze the setup without overly
  detailed knowledge of the surrogate loss. But they already assume
  the surrogate is lipschitz continuous and bounded. What more
  information would we need to conduct the analysis without taking the
  U-statistics route?}
The authors then prove a ULLN for the risk
functional they define and show that it is consistent for the
underlying listwise losses.

I still have not spent sufficient time on the experimental section to provide an overview.

\subsection{Proving a ULLN for the U-statistics Based Risk Functional}

The proof for the uniform law of large numbers for the risk functional \(\hat{R}_{n, \varphi}\) can be decomposed into controlling three terms:
...

\subsubsection{Potential improvements}
The U-statistics formulation is totally ignored in controlling term (I), leading to the typical rate.
\footnote{actually, this rate is not even \(O_p(1/\sqrt n)\) since the concentration inequality contains an additional term involving \(\log(N)\) where \(N\) is the \(\epsilon\)-covering number of the space of functions. Can we get rid of this?}

\end{document}

%%% Local Variables:
%%% mode: latex
%%% TeX-master: t
%%% End:
